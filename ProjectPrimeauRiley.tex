% Options for packages loaded elsewhere
\PassOptionsToPackage{unicode}{hyperref}
\PassOptionsToPackage{hyphens}{url}
%
\documentclass[
]{article}
\usepackage{amsmath,amssymb}
\usepackage{lmodern}
\usepackage{iftex}
\ifPDFTeX
  \usepackage[T1]{fontenc}
  \usepackage[utf8]{inputenc}
  \usepackage{textcomp} % provide euro and other symbols
\else % if luatex or xetex
  \usepackage{unicode-math}
  \defaultfontfeatures{Scale=MatchLowercase}
  \defaultfontfeatures[\rmfamily]{Ligatures=TeX,Scale=1}
\fi
% Use upquote if available, for straight quotes in verbatim environments
\IfFileExists{upquote.sty}{\usepackage{upquote}}{}
\IfFileExists{microtype.sty}{% use microtype if available
  \usepackage[]{microtype}
  \UseMicrotypeSet[protrusion]{basicmath} % disable protrusion for tt fonts
}{}
\makeatletter
\@ifundefined{KOMAClassName}{% if non-KOMA class
  \IfFileExists{parskip.sty}{%
    \usepackage{parskip}
  }{% else
    \setlength{\parindent}{0pt}
    \setlength{\parskip}{6pt plus 2pt minus 1pt}}
}{% if KOMA class
  \KOMAoptions{parskip=half}}
\makeatother
\usepackage{xcolor}
\usepackage[margin=1in]{geometry}
\usepackage{color}
\usepackage{fancyvrb}
\newcommand{\VerbBar}{|}
\newcommand{\VERB}{\Verb[commandchars=\\\{\}]}
\DefineVerbatimEnvironment{Highlighting}{Verbatim}{commandchars=\\\{\}}
% Add ',fontsize=\small' for more characters per line
\usepackage{framed}
\definecolor{shadecolor}{RGB}{248,248,248}
\newenvironment{Shaded}{\begin{snugshade}}{\end{snugshade}}
\newcommand{\AlertTok}[1]{\textcolor[rgb]{0.94,0.16,0.16}{#1}}
\newcommand{\AnnotationTok}[1]{\textcolor[rgb]{0.56,0.35,0.01}{\textbf{\textit{#1}}}}
\newcommand{\AttributeTok}[1]{\textcolor[rgb]{0.77,0.63,0.00}{#1}}
\newcommand{\BaseNTok}[1]{\textcolor[rgb]{0.00,0.00,0.81}{#1}}
\newcommand{\BuiltInTok}[1]{#1}
\newcommand{\CharTok}[1]{\textcolor[rgb]{0.31,0.60,0.02}{#1}}
\newcommand{\CommentTok}[1]{\textcolor[rgb]{0.56,0.35,0.01}{\textit{#1}}}
\newcommand{\CommentVarTok}[1]{\textcolor[rgb]{0.56,0.35,0.01}{\textbf{\textit{#1}}}}
\newcommand{\ConstantTok}[1]{\textcolor[rgb]{0.00,0.00,0.00}{#1}}
\newcommand{\ControlFlowTok}[1]{\textcolor[rgb]{0.13,0.29,0.53}{\textbf{#1}}}
\newcommand{\DataTypeTok}[1]{\textcolor[rgb]{0.13,0.29,0.53}{#1}}
\newcommand{\DecValTok}[1]{\textcolor[rgb]{0.00,0.00,0.81}{#1}}
\newcommand{\DocumentationTok}[1]{\textcolor[rgb]{0.56,0.35,0.01}{\textbf{\textit{#1}}}}
\newcommand{\ErrorTok}[1]{\textcolor[rgb]{0.64,0.00,0.00}{\textbf{#1}}}
\newcommand{\ExtensionTok}[1]{#1}
\newcommand{\FloatTok}[1]{\textcolor[rgb]{0.00,0.00,0.81}{#1}}
\newcommand{\FunctionTok}[1]{\textcolor[rgb]{0.00,0.00,0.00}{#1}}
\newcommand{\ImportTok}[1]{#1}
\newcommand{\InformationTok}[1]{\textcolor[rgb]{0.56,0.35,0.01}{\textbf{\textit{#1}}}}
\newcommand{\KeywordTok}[1]{\textcolor[rgb]{0.13,0.29,0.53}{\textbf{#1}}}
\newcommand{\NormalTok}[1]{#1}
\newcommand{\OperatorTok}[1]{\textcolor[rgb]{0.81,0.36,0.00}{\textbf{#1}}}
\newcommand{\OtherTok}[1]{\textcolor[rgb]{0.56,0.35,0.01}{#1}}
\newcommand{\PreprocessorTok}[1]{\textcolor[rgb]{0.56,0.35,0.01}{\textit{#1}}}
\newcommand{\RegionMarkerTok}[1]{#1}
\newcommand{\SpecialCharTok}[1]{\textcolor[rgb]{0.00,0.00,0.00}{#1}}
\newcommand{\SpecialStringTok}[1]{\textcolor[rgb]{0.31,0.60,0.02}{#1}}
\newcommand{\StringTok}[1]{\textcolor[rgb]{0.31,0.60,0.02}{#1}}
\newcommand{\VariableTok}[1]{\textcolor[rgb]{0.00,0.00,0.00}{#1}}
\newcommand{\VerbatimStringTok}[1]{\textcolor[rgb]{0.31,0.60,0.02}{#1}}
\newcommand{\WarningTok}[1]{\textcolor[rgb]{0.56,0.35,0.01}{\textbf{\textit{#1}}}}
\usepackage{graphicx}
\makeatletter
\def\maxwidth{\ifdim\Gin@nat@width>\linewidth\linewidth\else\Gin@nat@width\fi}
\def\maxheight{\ifdim\Gin@nat@height>\textheight\textheight\else\Gin@nat@height\fi}
\makeatother
% Scale images if necessary, so that they will not overflow the page
% margins by default, and it is still possible to overwrite the defaults
% using explicit options in \includegraphics[width, height, ...]{}
\setkeys{Gin}{width=\maxwidth,height=\maxheight,keepaspectratio}
% Set default figure placement to htbp
\makeatletter
\def\fps@figure{htbp}
\makeatother
\setlength{\emergencystretch}{3em} % prevent overfull lines
\providecommand{\tightlist}{%
  \setlength{\itemsep}{0pt}\setlength{\parskip}{0pt}}
\setcounter{secnumdepth}{-\maxdimen} % remove section numbering
\ifLuaTeX
  \usepackage{selnolig}  % disable illegal ligatures
\fi
\IfFileExists{bookmark.sty}{\usepackage{bookmark}}{\usepackage{hyperref}}
\IfFileExists{xurl.sty}{\usepackage{xurl}}{} % add URL line breaks if available
\urlstyle{same} % disable monospaced font for URLs
\hypersetup{
  hidelinks,
  pdfcreator={LaTeX via pandoc}}

\author{}
\date{\vspace{-2.5em}}

\begin{document}

\hypertarget{ch-6.3.1-multidimensional-gradient-descent}{%
\section{Ch 6.3.1: Multidimensional Gradient
Descent}\label{ch-6.3.1-multidimensional-gradient-descent}}

author: Riley Primeau date: `2022-12-07'\\
autosize: true

\hypertarget{gradient-descent}{%
\section{Gradient Descent}\label{gradient-descent}}

Gradient descent is used for optimization problems where you need to
find a maximum/minimum. Like the Newton-Raphson method for finding
roots, this method relies on knowing the first derivative and is an
iterative process.

\begin{itemize}
\tightlist
\item
  Finds derivative at x
\item
  Takes a step down/up of size h in the direction of the slope found at
  x
\item
  As the function approaches a peak/trough through each iteration, it
  will result in smaller changes in x
\item
  Once the change in x decreases below the tolerance level, a maximum or
  minimum has been reached
\end{itemize}

\hypertarget{gradient-descent-in-1-dimension}{%
\section{Gradient Descent in 1
Dimension}\label{gradient-descent-in-1-dimension}}

\hypertarget{gradient-descent-can-be-expanded-to-multiple-dimensions}{%
\section{Gradient Descent can be Expanded to Multiple
Dimensions}\label{gradient-descent-can-be-expanded-to-multiple-dimensions}}

\begin{itemize}
\tightlist
\item
  Altering the 1-dimensional gradient descent function to work with
  multiple dimensions is fairly simple
\item
  The variables for x and the derivative must be made into vectors to
  accommodate for more dimensions
\item
  Partial derivatives for each variable input are now needed
\end{itemize}

\hypertarget{implementation}{%
\section{Implementation}\label{implementation}}

\begin{Shaded}
\begin{Highlighting}[]
\NormalTok{gd }\OtherTok{=} \ControlFlowTok{function}\NormalTok{(fp, x, }\AttributeTok{h =} \FloatTok{1e2}\NormalTok{, }\AttributeTok{tol =} \FloatTok{1e{-}4}\NormalTok{, }\AttributeTok{m =} \FloatTok{1e3}\NormalTok{)}
\NormalTok{\{}
\NormalTok{  iter }\OtherTok{=} \DecValTok{0}
  
\NormalTok{  oldx }\OtherTok{=}\NormalTok{ x}
\NormalTok{  x }\OtherTok{=}\NormalTok{ x }\SpecialCharTok{{-}}\NormalTok{ h }\SpecialCharTok{*} \FunctionTok{fp}\NormalTok{(x)}
  
  \ControlFlowTok{while}\NormalTok{(}\FunctionTok{vecnorm}\NormalTok{(x }\SpecialCharTok{{-}}\NormalTok{ oldx) }\SpecialCharTok{\textgreater{}}\NormalTok{ tol)}
\NormalTok{  \{}
\NormalTok{    iter }\OtherTok{=}\NormalTok{ iter }\SpecialCharTok{+} \DecValTok{1}
    \ControlFlowTok{if}\NormalTok{(iter }\SpecialCharTok{\textgreater{}}\NormalTok{ m)}
      \FunctionTok{return}\NormalTok{(x)}
\NormalTok{    oldx }\OtherTok{=}\NormalTok{ x}
\NormalTok{    x }\OtherTok{=}\NormalTok{ x }\SpecialCharTok{{-}}\NormalTok{ h }\SpecialCharTok{*} \FunctionTok{fp}\NormalTok{(x)}
\NormalTok{  \}}
  \FunctionTok{return}\NormalTok{(x)}
\NormalTok{\}}
\end{Highlighting}
\end{Shaded}

\hypertarget{things-to-note}{%
\section{Things to note\ldots{}}\label{things-to-note}}

\begin{itemize}
\tightlist
\item
  fp is a vector of functions
\item
  x is a vector of values
\item
  The function vecnorm(x, p=2) finds the magnitude of the vector. The
  power p can be changed but the default value of p is 2. This is known
  as the Euclidean Norm
\end{itemize}

\hypertarget{more-important-aspects}{%
\section{More important aspects}\label{more-important-aspects}}

\begin{itemize}
\tightlist
\item
  Each variable will converge to their values at different rates
\item
  Increasing the step limit improves results but increases flops
\item
  Larger step size typically means faster convergence BUT too large of a
  step size can lead to unpredictable results
\end{itemize}

\hypertarget{analogy}{%
\section{Analogy}\label{analogy}}

A person is stuck in the mountains and is trying to get down (i.e.,
trying to find the global minimum). There is heavy fog such that
visibility is extremely low. Therefore, the path down the mountain is
not visible, so they must use local information to find the minimum.
They can use the method of gradient descent, which involves looking at
the steepness of the hill at their current position, then proceeding in
the direction with the steepest descent (i.e., downhill).

\hypertarget{d-visualizations}{%
\section{2D Visualizations}\label{d-visualizations}}

\hypertarget{applications-of-multidimensional-gradient-descent}{%
\section{Applications of Multidimensional Gradient
Descent}\label{applications-of-multidimensional-gradient-descent}}

\begin{itemize}
\tightlist
\item
  Training machine learning neural networks
\item
  \url{https://youtu.be/aircAruvnKk?t=334}
\item
  Gradient descent is used during back propagation in order to update
  weights to make more accurate predictions
\end{itemize}

\hypertarget{references}{%
\section{References}\label{references}}

\begin{itemize}
\tightlist
\item
  Howard, II, J.P. (2017). Computational Methods for Numerical Analysis
  with R (1st ed.). Chapman and Hall/CRC.
  \url{https://doi.org/10.1201/9781315120195}
\item
  Wikipedia
\item
  ``SGD.'' Hasty.ai, 16 Nov.~2022,
  \url{https://hasty.ai/docs/mp-wiki/solvers-optimizers/sgd}.
\item
  Jiang, Lili. ``A Visual Explanation of Gradient Descent Methods
  (Momentum, AdaGrad, RMSProp, Adam).'' Medium, Towards Data Science, 21
  Sept.~2020,
  \url{https://towardsdatascience.com/a-visual-explanation-of-gradient-descent-methods-momentum-adagrad-rmsprop-adam-f898b102325c}.
\end{itemize}

\end{document}
